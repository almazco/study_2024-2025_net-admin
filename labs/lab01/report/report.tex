% Options for packages loaded elsewhere
% Options for packages loaded elsewhere
\PassOptionsToPackage{unicode}{hyperref}
\PassOptionsToPackage{hyphens}{url}
%
\documentclass[
  english,
  russian,
  12pt,
  a4paper,
  DIV=11,
  numbers=noendperiod]{scrreprt}
\usepackage{xcolor}
\usepackage{amsmath,amssymb}
\setcounter{secnumdepth}{5}
\usepackage{iftex}
\ifPDFTeX
  \usepackage[T1]{fontenc}
  \usepackage[utf8]{inputenc}
  \usepackage{textcomp} % provide euro and other symbols
\else % if luatex or xetex
  \usepackage{unicode-math} % this also loads fontspec
  \defaultfontfeatures{Scale=MatchLowercase}
  \defaultfontfeatures[\rmfamily]{Ligatures=TeX,Scale=1}
\fi
\usepackage{lmodern}
\ifPDFTeX\else
  % xetex/luatex font selection
\fi
% Use upquote if available, for straight quotes in verbatim environments
\IfFileExists{upquote.sty}{\usepackage{upquote}}{}
\IfFileExists{microtype.sty}{% use microtype if available
  \usepackage[]{microtype}
  \UseMicrotypeSet[protrusion]{basicmath} % disable protrusion for tt fonts
}{}
\usepackage{setspace}
% Make \paragraph and \subparagraph free-standing
\makeatletter
\ifx\paragraph\undefined\else
  \let\oldparagraph\paragraph
  \renewcommand{\paragraph}{
    \@ifstar
      \xxxParagraphStar
      \xxxParagraphNoStar
  }
  \newcommand{\xxxParagraphStar}[1]{\oldparagraph*{#1}\mbox{}}
  \newcommand{\xxxParagraphNoStar}[1]{\oldparagraph{#1}\mbox{}}
\fi
\ifx\subparagraph\undefined\else
  \let\oldsubparagraph\subparagraph
  \renewcommand{\subparagraph}{
    \@ifstar
      \xxxSubParagraphStar
      \xxxSubParagraphNoStar
  }
  \newcommand{\xxxSubParagraphStar}[1]{\oldsubparagraph*{#1}\mbox{}}
  \newcommand{\xxxSubParagraphNoStar}[1]{\oldsubparagraph{#1}\mbox{}}
\fi
\makeatother


\usepackage{longtable,booktabs,array}
\usepackage{calc} % for calculating minipage widths
% Correct order of tables after \paragraph or \subparagraph
\usepackage{etoolbox}
\makeatletter
\patchcmd\longtable{\par}{\if@noskipsec\mbox{}\fi\par}{}{}
\makeatother
% Allow footnotes in longtable head/foot
\IfFileExists{footnotehyper.sty}{\usepackage{footnotehyper}}{\usepackage{footnote}}
\makesavenoteenv{longtable}
\usepackage{graphicx}
\makeatletter
\newsavebox\pandoc@box
\newcommand*\pandocbounded[1]{% scales image to fit in text height/width
  \sbox\pandoc@box{#1}%
  \Gscale@div\@tempa{\textheight}{\dimexpr\ht\pandoc@box+\dp\pandoc@box\relax}%
  \Gscale@div\@tempb{\linewidth}{\wd\pandoc@box}%
  \ifdim\@tempb\p@<\@tempa\p@\let\@tempa\@tempb\fi% select the smaller of both
  \ifdim\@tempa\p@<\p@\scalebox{\@tempa}{\usebox\pandoc@box}%
  \else\usebox{\pandoc@box}%
  \fi%
}
% Set default figure placement to htbp
\def\fps@figure{htbp}
\makeatother



\ifLuaTeX
\usepackage[bidi=basic,provide=*]{babel}
\else
\usepackage[bidi=default,provide=*]{babel}
\fi
% get rid of language-specific shorthands (see #6817):
\let\LanguageShortHands\languageshorthands
\def\languageshorthands#1{}


\setlength{\emergencystretch}{3em} % prevent overfull lines

\providecommand{\tightlist}{%
  \setlength{\itemsep}{0pt}\setlength{\parskip}{0pt}}



 
\usepackage[style=gost-numeric,backend=biber,langhook=extras,autolang=other*]{biblatex}
\addbibresource{bib/cite.bib}

\usepackage[]{csquotes}

\KOMAoption{captions}{tableheading}
\usepackage{indentfirst}
\usepackage{float}
\floatplacement{figure}{H}
\usepackage[math,RM={Scale=0.94},SS={Scale=0.94},SScon={Scale=0.94},TT={Scale=MatchLowercase,FakeStretch=0.9},DefaultFeatures={Ligatures=Common}]{plex-otf}
\makeatletter
\@ifpackageloaded{caption}{}{\usepackage{caption}}
\AtBeginDocument{%
\ifdefined\contentsname
  \renewcommand*\contentsname{Содержание}
\else
  \newcommand\contentsname{Содержание}
\fi
\ifdefined\listfigurename
  \renewcommand*\listfigurename{Список иллюстраций}
\else
  \newcommand\listfigurename{Список иллюстраций}
\fi
\ifdefined\listtablename
  \renewcommand*\listtablename{Список таблиц}
\else
  \newcommand\listtablename{Список таблиц}
\fi
\ifdefined\figurename
  \renewcommand*\figurename{Рисунок}
\else
  \newcommand\figurename{Рисунок}
\fi
\ifdefined\tablename
  \renewcommand*\tablename{Таблица}
\else
  \newcommand\tablename{Таблица}
\fi
}
\@ifpackageloaded{float}{}{\usepackage{float}}
\floatstyle{ruled}
\@ifundefined{c@chapter}{\newfloat{codelisting}{h}{lop}}{\newfloat{codelisting}{h}{lop}[chapter]}
\floatname{codelisting}{Список}
\newcommand*\listoflistings{\listof{codelisting}{Листинги}}
\makeatother
\makeatletter
\makeatother
\makeatletter
\@ifpackageloaded{caption}{}{\usepackage{caption}}
\@ifpackageloaded{subcaption}{}{\usepackage{subcaption}}
\makeatother
\usepackage{bookmark}
\IfFileExists{xurl.sty}{\usepackage{xurl}}{} % add URL line breaks if available
\urlstyle{same}
\hypersetup{
  pdftitle={Отчёт по лабораторной работе №1},
  pdfauthor={Зиязетдинов Алмаз Радикович},
  pdflang={ru-RU},
  hidelinks,
  pdfcreator={LaTeX via pandoc}}


\title{Отчёт по лабораторной работе №1}
\author{Зиязетдинов Алмаз Радикович}
\date{}
\begin{document}
\maketitle

\renewcommand*\contentsname{Содержание}
{
\setcounter{tocdepth}{1}
\tableofcontents
}
\listoffigures
\listoftables

\setstretch{1.5}
\chapter{Цель
работы}\label{ux446ux435ux43bux44c-ux440ux430ux431ux43eux442ux44b}

Установка инструмента моделирования конфигурации сети Cisco Packet
Tracer {[}3{]}, знакомство с его интерфейсом.

\chapter{Задание}\label{ux437ux430ux434ux430ux43dux438ux435}

\begin{enumerate}
\def\labelenumi{\arabic{enumi}.}
\tightlist
\item
  Установить на домашнем устройстве Cisco Packet Tracer.
\item
  Постройте простейшую сеть в Cisco Packet Tracer, проведите простейшую
  настройку оборудования.
\end{enumerate}

\chapter{Последовательность выполнения
работы}\label{ux43fux43eux441ux43bux435ux434ux43eux432ux430ux442ux435ux43bux44cux43dux43eux441ux442ux44c-ux432ux44bux43fux43eux43bux43dux435ux43dux438ux44f-ux440ux430ux431ux43eux442ux44b}

\begin{verbatim}
1. Установите в вашей операционной системе Cisco Packet Tracer     ([рис.1 @fig-001]).                                                                 |
\end{verbatim}

Построение простейшей сети 1. Создайте новый проект (например,
lab\_PT-01.pkt). 2. В рабочем пространстве разместите концентратор
(Hub-PT) и четыре оконечных устройства PC. Соедините оконечные
устройства с концентратором прямым кабелем Щёлкнув последовательно на
каждом оконечном устройстве, задайте статические IP-адреса 192.168.1.11,
192.168.1.12, 192.168.1.13, 192.168.1.14 с маской подсети 255.255.255.0

Модель простой сети с концентратором (\textbf{?@fig-001}). \textbar{}

\begin{enumerate}
\def\labelenumi{\arabic{enumi}.}
\setcounter{enumi}{2}
\tightlist
\item
  В основном окне проекта перейдите из режима реального времени
  (Realtime) в режим моделирования (Simulation). Выберите на панели
  инструментов мышкой «Add Simple PDU (P)» и щёлкните сначала на PC0,
  затем на PC2. В рабочей области должны будут появится два конверта,
  обозначающих пакеты, в списке событий на панели моделирования должны
  будут появиться два события, относящихся к пакетам ARP и ICMP
  соответственно (рис. 1.5). На панели моделирования нажмите кнопку
  «Play» и проследите за движением пакетов ARP и ICMP от устройства PC0
  до устройства PC2 и обратно.
\end{enumerate}

Простейшая модель сети (\textbf{?@fig-004}).\\
4. Щёлкнув на строке события, откройте окно информации о PDU и изучите,
что происходит на уровне модели OSI при перемещении пакета (рис. 1.6).
Используя кнопку «Проверь себя» (Challenge Me) на вкладке OSI Model,
ответьте на вопросы. 5. Откройте вкладку с информацией о PDU (рис. 1.7).
Исследуйте структуру пакета ICMP. Опишите структуру кадра Ethernet.
Какие изменения происходят в кадре Ethernet при передвижении пакета?
Какой тип имеет кадр Ethernet? Опишите структуру MAC-адресов. 6.
Очистите список событий, удалив сценарий моделирования. Выберите на
панели инструментов мышкой «Add Simple PDU (P)» и щёлкните сначала на
PC0, затем на PC2. Снова выберите на панели инструментов мышкой «Add
Simple PDU (P)» и щёлкните сначала на PC2, затем на PC0. На панели
моделирования нажмите кнопку «Play» и проследите за возникновением
коллизии (рис. 1.8). В списке событий посмотрите информацию о PDU. В
отчёте поясните, как отображается в заголовках пакетов информация о
коллизии и почему возникла коллизия.

События в режиме моделирования Packet Tracer(\textbf{?@fig-003}).

\begin{enumerate}
\def\labelenumi{\arabic{enumi}.}
\setcounter{enumi}{6}
\tightlist
\item
  Перейдите в режим реального времени (Realtime). В рабочем пространстве
  разместите коммутатор (например Cisco 2950-24) и 4 оконечных
  устройства PC. Соедините оконечные устройства с коммутатором прямым
  кабелем. Щёлкнув последовательно на каждом оконечном устройстве,
  задайте статические IP-адреса 192.168.1.21, 192.168.1.22,
  192.168.1.23, 192.168.1.24 с маской подсети 255.255.255.0.
\item
  В основном окне проекта перейдите из режима реального времени
  (Realtime) в режим моделирования (Simulation). Выберите на панели
  инструментов мышкой «Add Simple PDU (P)» и щёлкните сначала на PC4,
  затем на PC6. В рабочей области должны будут появится два конверта,
  обозначающих пакеты, в списке событий на панели моделирования должны
  будут появиться два события, относящихся к пакетам ARP и ICMP
  соответственно
\end{enumerate}

События в режиме моделирования Packet Tracer (\textbf{?@fig-005}). (рис.
1.9). На панели моделирования нажмите кнопку «Play» и проследите за
движением пакетов ARP и ICMP от устройства PC4 до устройства PC6 и
обратно. В отчёте поясните, есть ли различия и в чём они заключаются в
событиях протокола ARP в сценарии с концентратором.

Модель простой сети с коммутатором(\textbf{?@fig-006}).

\begin{verbatim}
                                                    |
\end{verbatim}

\begin{enumerate}
\def\labelenumi{\arabic{enumi}.}
\setcounter{enumi}{8}
\item
  Исследуйте структуру пакета ICMP. Опишите структуру кадра Ethernet.
  Какие изменения происходят в кадре Ethernet при передвижении пакета?
  Какой тип имеет кадр Ethernet? Опишите структуру MAC-адресов.
\item
  Очистите список событий, удалив сценарий моделирования. Выберите на
  панели инструментов мышкой «Add Simple PDU (P)» и щёлкните сначала на
  PC4, затем на PC6. Снова выберите на панели инструментов мышкой «Add
  Simple PDU (P)» и щёлкните сначала на PC6, затем на PC4. На панели
  моделирования нажмите кнопку «Play» и проследите за движением пакетов.
  В отчёте поясните, почему не возникает коллизия.
\item
  Перейдите в режим реального времени (Realtime). В рабочем пространстве
  соедините кроссовым кабелем концентратор и коммутатор. Перейдите в
  режим моделирования (Simulation). Очистите список событий, удалив
  сценарий моделирования. Выберите на панели инструментов мышкой «Add
  Simple PDU (P)» и щёлкните сначала на PC0, затем на PC4. Снова
  выберите на панели инструментов мышкой «Add Simple PDU (P)» и щёлкните
  сначала на PC4, затем на PC0. На панели моделирования нажмите кнопку
  «Play» и проследите за движением пакетов. В отчёте поясните, почему
  сначала возникает коллизия (рис. 1.10), а затем пакеты успешно
  достигают пункта назначения.
\item
  Очистите список событий, удалив сценарий моделирования. На панели
  моделирования нажмите «Play» и в списке событий получите пакеты STP
  (рис. 1.11). Исследуйте структуру STP. Опишите структуру кадра
  Ethernet в этих пакетах. Какой тип имеет кадр Ethernet? Опишите
  структуру MACадресов.
\item
  Перейдите в режим реального времени (Realtime). В рабочем пространстве
  добавьте маршрутизатор (например, Cisco 2811). Соедините прямым
  кабелем коммутатор и маршрутизатор (рис. 1.12). Щёлкните на
  маршрутизаторе и на вкладке его конфигурации пропишите статический
  IP-адрес 192.168.1.254 с маской 255.255.255.0, активируйте порт,
  поставив галочку «On» напротив «Port Status» (рис. 1.13).
\item
  Перейдите в режим моделирования (Simulation). Очистите список событий,
  удалив сценарий моделирования. Выберите на панели инструментов мышкой
  «Add Simple PDU (P)» и щёлкните сначала на PC3, затем на
  маршрутизаторе. На панели моделирования нажмите кнопку «Play» и
  проследите за движением пакетов ARP, ICMP, STP и CDP. Исследуйте
  структуру пакета CDP, опишите структуру кадра Ethernet. Какой тип
  имеет кадр Ethernet? Опишите структуру MAC-адресов.
\end{enumerate}

Модель простой сети с маршрутизатором(\textbf{?@fig-007}).\\
Сценарий с протоколом STP (\textbf{?@fig-008}).

\chapter{Выводы}\label{ux432ux44bux432ux43eux434ux44b}

Вывод: В ходе выполнения лабораторной работы мы научились устанавливать
инструмент моделирования конфигурации сети Cisco Packet Tracer без
учётной записи и познакомились с его интерфейсом.

\chapter{Ответы на контрольные
вопросы:}\label{ux43eux442ux432ux435ux442ux44b-ux43dux430-ux43aux43eux43dux442ux440ux43eux43bux44cux43dux44bux435-ux432ux43eux43fux440ux43eux441ux44b}

1 Дайте определение следующим понятиям: концентратор, коммутатор,
маршрутизатор, шлюз (gateway). В каких случаях следует использовать тот
или иной тип сетевого оборудования? Концентратор (Hub): концентратор
является устройством, которое принимает данные с одного устройства сети
и передает их всем остальным устройствам в сети. Он работает на
физическом уровне модели OSI (Open Systems Interconnection), просто
усиливая сигнал и передавая его по всем портам. Концентратор не имеет
интеллекта для анализа данных или управления трафиком. Обычно
используется в небольших сетях или для расширения количества портов в
сети. Коммутатор (Switch): коммутатор также работает на канальном уровне
OSI и способен анализировать адреса MAC (Media Access Control)
устройств, подключенных к нему. В отличие от концентратора, коммутатор
передает данные только тому устройству, для которого они предназначены,
что делает его более эффективным по сравнению с концентратором.

\printbibliography[heading=none]





\end{document}
